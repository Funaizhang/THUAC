%%%%%%%%%%%%%%%%%%%%%%%%%%%%%%%%%%%%%%%%%
% Lachaise Assignment
% LaTeX Template
% Version 1.0 (26/6/2018)
%
% This template originates from:
% http://www.LaTeXTemplates.com
%
% Authors:
% Marion Lachaise & François Févotte
% Vel (vel@LaTeXTemplates.com)
%
% License:
% CC BY-NC-SA 3.0 (http://creativecommons.org/licenses/by-nc-sa/3.0/)
% 
%%%%%%%%%%%%%%%%%%%%%%%%%%%%%%%%%%%%%%%%%

%----------------------------------------------------------------------------------------
%	PACKAGES AND OTHER DOCUMENT CONFIGURATIONS
%----------------------------------------------------------------------------------------

\documentclass{article}

%%%%%%%%%%%%%%%%%%%%%%%%%%%%%%%%%%%%%%%%%
% Lachaise Assignment
% Structure Specification File
% Version 1.0 (26/6/2018)
%
% This template originates from:
% http://www.LaTeXTemplates.com
%
% Authors:
% Marion Lachaise & François Févotte
% Vel (vel@LaTeXTemplates.com)
%
% License:
% CC BY-NC-SA 3.0 (http://creativecommons.org/licenses/by-nc-sa/3.0/)
% 
%%%%%%%%%%%%%%%%%%%%%%%%%%%%%%%%%%%%%%%%%

%----------------------------------------------------------------------------------------
%	PACKAGES AND OTHER DOCUMENT CONFIGURATIONS
%----------------------------------------------------------------------------------------

\usepackage{amsmath,amsfonts,stmaryrd,amssymb} % Math packages

\usepackage{enumerate} % Custom item numbers for enumerations

\usepackage[ruled]{algorithm2e,caption} % Algorithms

\usepackage[framemethod=tikz]{mdframed} % Allows defining custom boxed/framed environments

\usepackage{listings} % File listings, with syntax highlighting
\lstset{
	basicstyle=\ttfamily, % Typeset listings in monospace font
}

%----------------------------------------------------------------------------------------
%	DOCUMENT MARGINS
%----------------------------------------------------------------------------------------

\usepackage{geometry} % Required for adjusting page dimensions and margins

\geometry{
	paper=a4paper, % Paper size, change to letterpaper for US letter size
	top=2.5cm, % Top margin
	bottom=3cm, % Bottom margin
	left=2.5cm, % Left margin
	right=2.5cm, % Right margin
	headheight=14pt, % Header height
	footskip=1.5cm, % Space from the bottom margin to the baseline of the footer
	headsep=1.2cm, % Space from the top margin to the baseline of the header
	%showframe, % Uncomment to show how the type block is set on the page
}

%----------------------------------------------------------------------------------------
%	FONTS
%----------------------------------------------------------------------------------------

\usepackage[utf8]{inputenc} % Required for inputting international characters
\usepackage[T1]{fontenc} % Output font encoding for international characters

\usepackage{XCharter} % Use the XCharter fonts

%----------------------------------------------------------------------------------------
%	COMMAND LINE ENVIRONMENT
%----------------------------------------------------------------------------------------

% Usage:
% \begin{commandline}
%	\begin{verbatim}
%		$ ls
%		
%		Applications	Desktop	...
%	\end{verbatim}
% \end{commandline}

\mdfdefinestyle{commandline}{
	leftmargin=10pt,
	rightmargin=10pt,
	innerleftmargin=15pt,
	middlelinecolor=black!50!white,
	middlelinewidth=2pt,
	frametitlerule=false,
	backgroundcolor=black!5!white,
	frametitle={Sample Output},
	frametitlefont={\normalfont\sffamily\color{white}\hspace{-1em}},
	frametitlebackgroundcolor=black!50!white,
	nobreak,
}

% Define a custom environment for command-line snapshots
\newenvironment{commandline}{
	\medskip
	\begin{mdframed}[style=commandline]
}{
	\end{mdframed}
	\medskip
}

%----------------------------------------------------------------------------------------
%	FILE CONTENTS ENVIRONMENT
%----------------------------------------------------------------------------------------

% Usage:
% \begin{file}[optional filename, defaults to "File"]
%	File contents, for example, with a listings environment
% \end{file}

\mdfdefinestyle{file}{
	innertopmargin=1.6\baselineskip,
	innerbottommargin=0.8\baselineskip,
	topline=false, bottomline=false,
	leftline=false, rightline=false,
	leftmargin=2cm,
	rightmargin=2cm,
	singleextra={%
		\draw[fill=black!10!white](P)++(0,-1.2em)rectangle(P-|O);
		\node[anchor=north west]
		at(P-|O){\ttfamily\mdfilename};
		%
		\def\l{3em}
		\draw(O-|P)++(-\l,0)--++(\l,\l)--(P)--(P-|O)--(O)--cycle;
		\draw(O-|P)++(-\l,0)--++(0,\l)--++(\l,0);
	},
	nobreak,
}

% Define a custom environment for file contents
\newenvironment{file}[1][File]{ % Set the default filename to "File"
	\medskip
	\newcommand{\mdfilename}{#1}
	\begin{mdframed}[style=file]
}{
	\end{mdframed}
	\medskip
}

%----------------------------------------------------------------------------------------
%	NUMBERED QUESTIONS ENVIRONMENT
%----------------------------------------------------------------------------------------

% Usage:
% \begin{question}[optional title]
%	Question contents
% \end{question}

\mdfdefinestyle{question}{
	innertopmargin=1.2\baselineskip,
	innerbottommargin=0.8\baselineskip,
	roundcorner=5pt,
	nobreak,
	singleextra={%
		\draw(P-|O)node[xshift=1em,anchor=west,fill=white,draw,rounded corners=5pt]{%
		Question \theQuestion\questionTitle};
	},
}

\newcounter{Question} % Stores the current question number that gets iterated with each new question

% Define a custom environment for numbered questions
\newenvironment{question}[1][\unskip]{
	\bigskip
	\stepcounter{Question}
	\newcommand{\questionTitle}{~#1}
	\begin{mdframed}[style=question]
}{
	\end{mdframed}
	\medskip
}

%----------------------------------------------------------------------------------------
%	WARNING TEXT ENVIRONMENT
%----------------------------------------------------------------------------------------

% Usage:
% \begin{warn}[optional title, defaults to "Warning:"]
%	Contents
% \end{warn}

\mdfdefinestyle{warning}{
	topline=false, bottomline=false,
	leftline=false, rightline=false,
	nobreak,
	singleextra={%
		\draw(P-|O)++(-0.5em,0)node(tmp1){};
		\draw(P-|O)++(0.5em,0)node(tmp2){};
		\fill[black,rotate around={45:(P-|O)}](tmp1)rectangle(tmp2);
		\node at(P-|O){\color{white}\scriptsize\bf !};
		\draw[very thick](P-|O)++(0,-1em)--(O);%--(O-|P);
	}
}

% Define a custom environment for warning text
\newenvironment{warn}[1][Warning:]{ % Set the default warning to "Warning:"
	\medskip
	\begin{mdframed}[style=warning]
		\noindent{\textbf{#1}}
}{
	\end{mdframed}
}

%----------------------------------------------------------------------------------------
%	INFORMATION ENVIRONMENT
%----------------------------------------------------------------------------------------

% Usage:
% \begin{info}[optional title, defaults to "Info:"]
% 	contents
% 	\end{info}

\mdfdefinestyle{info}{%
	topline=false, bottomline=false,
	leftline=false, rightline=false,
	nobreak,
	singleextra={%
		\fill[black](P-|O)circle[radius=0.4em];
		\node at(P-|O){\color{white}\scriptsize\bf i};
		\draw[very thick](P-|O)++(0,-0.8em)--(O);%--(O-|P);
	}
}

% Define a custom environment for information
\newenvironment{info}[1][Trivia:]{ % Set the default title to "Info:"
	\medskip
	\begin{mdframed}[style=info]
		\noindent{\textbf{#1}}
}{
	\end{mdframed}
}
 % Include the file specifying the document structure and custom commands

%----------------------------------------------------------------------------------------
%	ASSIGNMENT INFORMATION
%----------------------------------------------------------------------------------------

\title{Project 1 Report: Cellphone Passcodes} % Title of the assignment

\author{Zhang Naifu\\ \texttt{znf18@mails.tsinghua.edu.cn}} % Author name and email address

\date{7 October 2018} % University, school and/or department name(s) and a date

%----------------------------------------------------------------------------------------

\begin{document}

\maketitle % Print the title

%----------------------------------------------------------------------------------------
%	Part 1
%----------------------------------------------------------------------------------------

\section{Results Summary} % Numbered section

We assume iPhone unlock consists of 4 keys with repetition from \{0,...,9\}, and that Android unlock screen is a 3x3 grid.

There are trivially 10\textsuperscript{4} = \textbf{10000} permutations for the iPhone unlock sequence.
The python script outputs \textbf{389112} permutations for the Android unlock sequence.
Assuming that the iPhone and Android users use all possible permutations of the unlock sequence with uniform probability, \textbf{Android is more secure}.


%----------------------------------------------------------------------------------------
%	Part 2
%----------------------------------------------------------------------------------------

\section{Implementation} % Numbered section

The python script is made up of 3 functions, \textit{isValid()}, \textit{bruteForce()} and \textit{semiBruteForce()}. 
\textit{isValid()} is used as a helper function in \textit{bruteForce()} and \textit{semiBruteForce()} to check for validity. \textit{semiBruteForce()} is a faster and more efficient version of \textit{bruteForce()} to enumerate permutations.

\subsection{Checking Validity} % Numbered subsection

Given a sequence, \textit{isValid()} returns \textit{True} if sequence is valid. Such a sequence must necessarily contain:
	\begin{enumerate} 
		\itemsep-0.3em 
		\item No repetition of any digits and
		\item No jumps (e.g. from 1 to 3) or 
		\item If there’s any jump, the jumped number in the middle must have appeared in the sequence before the jump
	\end{enumerate}
    
\subsection{Brute Force Enumeration} % Numbered subsection

\textit{bruteForce()} generates all possible permutations of length \textit{n}, including the invalid ones, before passing each through the \textit{isValid()} function. It gives a final count of the valid sequences.

To tally the total number of unlock sequences, \textit{bruteForce()} has to be called 6 times for \textit{n} = \{4,5,6,7,8,9\}, and takes \textasciitilde9 seconds. This was stretching my patience. 

\subsection{Exploiting Symmetries} % Numbered subsection
\textit{semiBruteForce()} implements some shortcuts.
	\begin{enumerate} 
	\itemsep-0.3em 
		\item The number of valid sequences starting from 1 is the same as that starting from any corner \{1,3,7,9\}; instead of generating all permutations of length \textit{n}, \textit{semiBruteForce()} only considers those starting with 1, and count every such valid unlock sequence 4 times. By the same token, this is true of the set \{2,4,6,8\} as well
		\item \textit{semiBruteForce(8)} = \textit{semiBruteForce(9)} for obvious reason
	\end{enumerate}

%----------------------------------------------------------------------------------------
%	Part 3
%----------------------------------------------------------------------------------------

\section{Complexity} % Numbered section

The \textbf{time complexity} of \textit{isValid()} depends on \textit{str.count()} and \textit{str.find()}. 
\textit{str.count()} implements fastsearch and is therefore \textit{O(n)}. 
\textit{str.find()} implements a mix between Boyer-More and Horspool, which is \textit{O(n)} on average. 

\textit{bruteForce()} calls on \textit{itertools.permutations()} with \textit{O(n!)} once, and \textit{isValid() n!} times. Abusing the notation a little, the total running time from length 4 to 9 is
\[\sum\limits_{n=4}^9 O(n!)+n!O(n)\]

\textit{semiBruteForce()}  calls on \textit{itertools.permutations()} once and only calls on \textit{isValid()} for a subset of the permutations. The total running time is
\[\sum\limits_{n=4}^8 O(n!)+3(n-1)!O(n)\]

The last term in each equation asymptotically dominates. Although both have the same asymptotic efficiency, \textit{semiBruteForce()} is faster by a factor of \textasciitilde3 in practice, as it only considers permutations starting with \{1,2,5\} and we omit \textit{semiBruteForce(9)}. Refer to sample output below.

For the same reason, \textit{semiBruteForce()} is better in terms of \textbf{space complexity} by a factor of \textasciitilde3. 

%----------------------------------------------------------------------------------------
% Sample output "screenshot"
\begin{commandline}
	\begin{verbatim}
    
		bruteForce:  
		389112 different unlock sequences of length 4 to 9 
		9.07 seconds taken 
		semiBruteForce:  
		389112 different unlock sequences of length 4 to 9 
		2.57 seconds taken
        
	\end{verbatim}
\end{commandline}
%----------------------------------------------------------------------------------------

\subsection{Efficiency Improvements} % Numbered subsection
Further improvements could perhaps be made by considering only valid sequences of length \textit{n} when generating sequences of length \textit{n+1}, since no invalid sequence of length n would give valid sequence of length \textit{n+1}. \textbf{Dynamic programming} could also be used. But I’m happy with the 2.5 seconds \textit{semiBruteForce()} implementation so we stop here.

%----------------------------------------------------------------------------------------
%	References
%----------------------------------------------------------------------------------------
\section*{References} % Unnumbered section

python string documentation \textit{https://github.com/python/cpython/tree/v3.6.5/Objects/stringlib}\\
python itertools documentation \textit{https://docs.python.org/3.5/library/itertools.html\#itertools.permutations}

\end{document}


