%%%%%%%%%%%%%%%%%%%%%%%%%%%%%%%%%%%%%%%%%
% Lachaise Assignment
% LaTeX Template
% Version 1.0 (26/6/2018)
%
% This template originates from:
% http://www.LaTeXTemplates.com
%
% Authors:
% Marion Lachaise & François Févotte
% Vel (vel@LaTeXTemplates.com)
%
%%%%%%%%%%%%%%%%%%%%%%%%%%%%%%%%%%%%%%%%%
% Lachaise Assignment
% LaTeX Template
% Version 1.0 (26/6/2018)
%
% This template originates from:
% http://www.LaTeXTemplates.com
%
% Authors:
% Marion Lachaise & François Févotte
% Vel (vel@LaTeXTemplates.com)
%
% License:
% CC BY-NC-SA 3.0 (http://creativecommons.org/licenses/by-nc-sa/3.0/)
% 
%%%%%%%%%%%%%%%%%%%%%%%%%%%%%%%%%%%%%%%%%

%----------------------------------------------------------------------------------------
%	PACKAGES AND OTHER DOCUMENT CONFIGURATIONS
%----------------------------------------------------------------------------------------

\documentclass{article}

%%%%%%%%%%%%%%%%%%%%%%%%%%%%%%%%%%%%%%%%%
% Lachaise Assignment
% Structure Specification File
% Version 1.0 (26/6/2018)
%
% This template originates from:
% http://www.LaTeXTemplates.com
%
% Authors:
% Marion Lachaise & François Févotte
% Vel (vel@LaTeXTemplates.com)
%
% License:
% CC BY-NC-SA 3.0 (http://creativecommons.org/licenses/by-nc-sa/3.0/)
% 
%%%%%%%%%%%%%%%%%%%%%%%%%%%%%%%%%%%%%%%%%

%----------------------------------------------------------------------------------------
%	PACKAGES AND OTHER DOCUMENT CONFIGURATIONS
%----------------------------------------------------------------------------------------

\usepackage{amsmath,amsfonts,stmaryrd,amssymb} % Math packages

\usepackage{enumerate} % Custom item numbers for enumerations

\usepackage[ruled]{algorithm2e,caption} % Algorithms

\usepackage[framemethod=tikz]{mdframed} % Allows defining custom boxed/framed environments

\usepackage{listings} % File listings, with syntax highlighting
\lstset{
	basicstyle=\ttfamily, % Typeset listings in monospace font
}

%----------------------------------------------------------------------------------------
%	DOCUMENT MARGINS
%----------------------------------------------------------------------------------------

\usepackage{geometry} % Required for adjusting page dimensions and margins

\geometry{
	paper=a4paper, % Paper size, change to letterpaper for US letter size
	top=2.5cm, % Top margin
	bottom=3cm, % Bottom margin
	left=2.5cm, % Left margin
	right=2.5cm, % Right margin
	headheight=14pt, % Header height
	footskip=1.5cm, % Space from the bottom margin to the baseline of the footer
	headsep=1.2cm, % Space from the top margin to the baseline of the header
	%showframe, % Uncomment to show how the type block is set on the page
}

%----------------------------------------------------------------------------------------
%	FONTS
%----------------------------------------------------------------------------------------

\usepackage[utf8]{inputenc} % Required for inputting international characters
\usepackage[T1]{fontenc} % Output font encoding for international characters

\usepackage{XCharter} % Use the XCharter fonts

%----------------------------------------------------------------------------------------
%	COMMAND LINE ENVIRONMENT
%----------------------------------------------------------------------------------------

% Usage:
% \begin{commandline}
%	\begin{verbatim}
%		$ ls
%		
%		Applications	Desktop	...
%	\end{verbatim}
% \end{commandline}

\mdfdefinestyle{commandline}{
	leftmargin=10pt,
	rightmargin=10pt,
	innerleftmargin=15pt,
	middlelinecolor=black!50!white,
	middlelinewidth=2pt,
	frametitlerule=false,
	backgroundcolor=black!5!white,
	frametitle={Sample Output},
	frametitlefont={\normalfont\sffamily\color{white}\hspace{-1em}},
	frametitlebackgroundcolor=black!50!white,
	nobreak,
}

% Define a custom environment for command-line snapshots
\newenvironment{commandline}{
	\medskip
	\begin{mdframed}[style=commandline]
}{
	\end{mdframed}
	\medskip
}

%----------------------------------------------------------------------------------------
%	FILE CONTENTS ENVIRONMENT
%----------------------------------------------------------------------------------------

% Usage:
% \begin{file}[optional filename, defaults to "File"]
%	File contents, for example, with a listings environment
% \end{file}

\mdfdefinestyle{file}{
	innertopmargin=1.6\baselineskip,
	innerbottommargin=0.8\baselineskip,
	topline=false, bottomline=false,
	leftline=false, rightline=false,
	leftmargin=2cm,
	rightmargin=2cm,
	singleextra={%
		\draw[fill=black!10!white](P)++(0,-1.2em)rectangle(P-|O);
		\node[anchor=north west]
		at(P-|O){\ttfamily\mdfilename};
		%
		\def\l{3em}
		\draw(O-|P)++(-\l,0)--++(\l,\l)--(P)--(P-|O)--(O)--cycle;
		\draw(O-|P)++(-\l,0)--++(0,\l)--++(\l,0);
	},
	nobreak,
}

% Define a custom environment for file contents
\newenvironment{file}[1][File]{ % Set the default filename to "File"
	\medskip
	\newcommand{\mdfilename}{#1}
	\begin{mdframed}[style=file]
}{
	\end{mdframed}
	\medskip
}

%----------------------------------------------------------------------------------------
%	NUMBERED QUESTIONS ENVIRONMENT
%----------------------------------------------------------------------------------------

% Usage:
% \begin{question}[optional title]
%	Question contents
% \end{question}

\mdfdefinestyle{question}{
	innertopmargin=1.2\baselineskip,
	innerbottommargin=0.8\baselineskip,
	roundcorner=5pt,
	nobreak,
	singleextra={%
		\draw(P-|O)node[xshift=1em,anchor=west,fill=white,draw,rounded corners=5pt]{%
		Question \theQuestion\questionTitle};
	},
}

\newcounter{Question} % Stores the current question number that gets iterated with each new question

% Define a custom environment for numbered questions
\newenvironment{question}[1][\unskip]{
	\bigskip
	\stepcounter{Question}
	\newcommand{\questionTitle}{~#1}
	\begin{mdframed}[style=question]
}{
	\end{mdframed}
	\medskip
}

%----------------------------------------------------------------------------------------
%	WARNING TEXT ENVIRONMENT
%----------------------------------------------------------------------------------------

% Usage:
% \begin{warn}[optional title, defaults to "Warning:"]
%	Contents
% \end{warn}

\mdfdefinestyle{warning}{
	topline=false, bottomline=false,
	leftline=false, rightline=false,
	nobreak,
	singleextra={%
		\draw(P-|O)++(-0.5em,0)node(tmp1){};
		\draw(P-|O)++(0.5em,0)node(tmp2){};
		\fill[black,rotate around={45:(P-|O)}](tmp1)rectangle(tmp2);
		\node at(P-|O){\color{white}\scriptsize\bf !};
		\draw[very thick](P-|O)++(0,-1em)--(O);%--(O-|P);
	}
}

% Define a custom environment for warning text
\newenvironment{warn}[1][Warning:]{ % Set the default warning to "Warning:"
	\medskip
	\begin{mdframed}[style=warning]
		\noindent{\textbf{#1}}
}{
	\end{mdframed}
}

%----------------------------------------------------------------------------------------
%	INFORMATION ENVIRONMENT
%----------------------------------------------------------------------------------------

% Usage:
% \begin{info}[optional title, defaults to "Info:"]
% 	contents
% 	\end{info}

\mdfdefinestyle{info}{%
	topline=false, bottomline=false,
	leftline=false, rightline=false,
	nobreak,
	singleextra={%
		\fill[black](P-|O)circle[radius=0.4em];
		\node at(P-|O){\color{white}\scriptsize\bf i};
		\draw[very thick](P-|O)++(0,-0.8em)--(O);%--(O-|P);
	}
}

% Define a custom environment for information
\newenvironment{info}[1][Trivia:]{ % Set the default title to "Info:"
	\medskip
	\begin{mdframed}[style=info]
		\noindent{\textbf{#1}}
}{
	\end{mdframed}
}
 % Include the file specifying the document structure and custom commands
\usepackage{hyperref}
\usepackage{amsmath}
\newtheorem{hyp}{Hypothesis}
\renewcommand{\thealgocf}{}

\usepackage{float}
\restylefloat{table}

\usepackage{tikz}
\usetikzlibrary{calc, intersections, arrows, shapes, positioning, fit, backgrounds}

\usepackage{pgfplots}
%----------------------------------------------------------------------------------------
%	ASSIGNMENT INFORMATION
%----------------------------------------------------------------------------------------

\title{Project 3 Report: Finding Integer Partition Number} % Title of the assignment

\author{Zhang Naifu\\ \texttt{znf18@mails.tsinghua.edu.cn}\\ \texttt{funaizhang@github}}% Author name and email address

\date{Tsinghua University\\ \today} % University, school and/or department name(s) and a date

%----------------------------------------------------------------------------------------

\begin{document}

\maketitle % Print the title

%----------------------------------------------------------------------------------------
%	INTRODUCTION
%----------------------------------------------------------------------------------------

\section*{Abstract} % Unnumbered section

Project 3 aims to compute the number of possible integer partitions for a natural number input by the user. This is the number of distinct ways of representing \(n\) as a sum of natural numbers (with order irrelevant). The function to generate such integer partition numbers is called the partition function, denoted by \(p(n)\). 

\textit{calc\_partition\_number(n)} in \textit{Project\_3.py} implements a version of Euler's recursive formula using his \textbf{pentagonal number theorem} and \textbf{dynamic programming} methods. Implementation details are discussed in Section 3. We find this is reasonably fast for integers less than \(10000\). Efficiency is discussed in Section 4.

This project makes no attempt at enumerating each partition.

The proofs of certain key mathematical theorems and statements go beyond the scope of the course, or the ability of the author, and they might thus be used without proof.

%----------------------------------------------------------------------------------------
%	SECTION 1
%----------------------------------------------------------------------------------------

\section{Illustrative Examples}

% Numbered question, with an optional title
\begin{question}[\itshape]
There are five partitions for natural number 4:\\
\(4\)\\
\(3+1\)\\
\(2+2\)\\
\(2+1+1\)\\
\(1+1+1+1\)
\end{question}
% Numbered question, with an optional title
\begin{question}[\itshape]
By convention \(p(0)=1\), \(p(n)=0\) for negative \(n\).
\end{question}

%----------------------------------------------------------------------------------------
%	SECTION 2
%----------------------------------------------------------------------------------------

\section{Brute Force Enumeration} % Numbered section

An obvious way is to enumerate all the partitions of \(n\). This could be made memory efficient, but is clearly extremely asymptotically time inefficient.

The partition number for large \(n\) is approximated by the Hardy-Ramanujan estimate:\\
\[\lim_{n \to \infty} p(n) = \frac{1}{4n\sqrt{3}} \exp(\pi \sqrt{\frac{2n}{3}})\]

In other words, the best we can do is not good enough,
\[p(n) = \Omega (\exp(\alpha  \sqrt{n}))  \;\;\;\;\; \mathrm{where} \;  \alpha = \pi \sqrt{\frac{2}{3}}\]

The \(n^{-1}\) term disappears from the above time complexity equation because each partition does not take \(\Theta(1)\) time to generate. Each component integer in a partition does - so each partition takes \(\Theta(n)\) time.


%----------------------------------------------------------------------------------------
%	SECTION 3
%----------------------------------------------------------------------------------------

\section{Recurrence}

Instead of counting, we use Euler's recurrence relation to calculate \(p(n)\).

To begin with, we know the generating function for \(p(n)\):
\[G(x) =\sum_{n=0}^{\infty}p(n) x^{n}= (1+x+x^2+...)(1+x^2+x^4+...)(1+x^3+...)... = \prod_{n=1}^{\infty}\frac{1}{1-x^n}\]

Applying Euler's pentagonal number theorem without proof, the denominator could be expressed recursively,
\[\prod_{n=1}^{\infty}(1-x^n) = \sum_{k=-\infty}^{\infty} (-1)^k x^{3(k-1)/2} = 1+ \sum_{k=1}^{\infty} (-1)^k (x^{3(k-1)/2} + x^{3(k+1)/2})\]

We can then derive the recurrence relation that corresponds to this generating function using the method taught in class. Omitting the full derivation steps, the resulting recurrence relation is:
\[p(n) = \sum_{k=1}^{n} (-1)^{k+1}(p(n-\frac{3k(k-1)}{2}) + p(n-\frac{3k(k+1)}{2}))\]

The last equation is implemented in \textit{calc\_partition\_number(n)}. With \(n=1000\), we have the following output after a couple of seconds. For \(n=10000\), the programme would run for minutes.
% Command-line "screenshot"

\begin{center}
	\begin{minipage}{0.7\linewidth}
    	\begin{commandline}
			\begin{verbatim}
		$ ./Project_3.py
        
Please enter a natural number: 1000
p(1000) = 24061467864032622473692149727991      
			\end{verbatim}
		\end{commandline}
	\end{minipage}
\end{center}


%----------------------------------------------------------------------------------------
%	SECTION 3
%----------------------------------------------------------------------------------------

\section{Complexity}

\textit{calc\_partition\_number(n)} contains two loops, one nested in the other. The inner loop over \(k\) implements the recursive formula, while the outer loop over \(n\) iterates over each natural number up to \(n\) for memoization. Therefore the time complexity with dynamic programming is, 
\[p(n) = O(n^2)\]

This compares favorably with enumeration time complexity of \(\Omega (\exp(\alpha  \sqrt{n})) \).





Memoization takes up \(O(n)\) space - a small price to pay for the improvement in speed.

Admittedly, there are more efficient solutions but these tend to trade off elegance and readability for efficiency. The author has on this occasion opted for more elegance code with marginally worse efficiency.

%----------------------------------------------------------------------------------------
%	References
%----------------------------------------------------------------------------------------
\section*{References} % Unnumbered section

Wikipedia \textit{https://en.wikipedia.org/wiki/Partition\_(number\_theory)}\\
Wikipedia \textit{https://en.wikipedia.org/wiki/Pentagonal\_number\_theorem}\\
H. Wilf. \textit{Lectures on Integer Partitions.} University of Pennsylvania. 2000. \textit{https://www.math.upenn.edu/~wilf/PIMS/PIMSLectures.pdf}

%----------------------------------------------------------------------------------------

\end{document}
